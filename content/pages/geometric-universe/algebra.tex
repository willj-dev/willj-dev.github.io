------------------------
title: Geometric Algebra
------------------------

Alright, we have put off the math long enough!

\section{Fundamentals}

We'll start 

\section{Scalar Algebra; Vector Algebra; Complex Numbers and Beyond}

- Using only scalars gives us high school algebra (the algebra of a field over real numbers)

- Using only vectors gives us linear algebra, and in three dimensions we can recover dot/cross products

- Pseudoscalar as the ``imaginary unit''; multivectors as an extension of encoding complex numbers as a + bi

\section{Symmetry and Invariance}

- Definitions

- Symmetry Groups

- Conformal Symmetry

- Conformal Geometry of Spacetime

\section{Geometric Calculus}

- Limit; derivative; integral

- Stokes; FTC and other special cases

\section{Calculus of Variations and the Euler-Lagrange Equations}`'

``Solutions'' in GP are specifically solutions to the Euler-Lagrange equations, which are generally coupled partial differential equations and, in practice, almost always impossible to solve directly. This makes the GUH, in one sense, disappointing: what's the point of using math to describe physical laws if all we get out of it is equations we can't solve? But really it's not as bad as all that. In fact, there are many ways to information out of these equations without actually solving them; this is essentially what theoretical physics is, and recently we have also been able to tackle them numerically using computer algorithms. Besides, the simple cases that we \emph{can} solve exactly are still valuable for human understanding. We are in a sense doing the same thing humans usually do when faced with a complicated math problem: break it down into easily digestible chunks, and use mathematical notation to keep track of all the little bits so that we don't get overwhelmed.

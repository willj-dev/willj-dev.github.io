------------------------
title: The Geometric Universe Hypothesis
------------------------

\begin{description}
  \item[The Geometric Universe Hypothesis] The \emph{stationary action principle} suffices to explain all observed phenomena under perfectly controlled conditions. We have no idea why.
\end{description}

This is enough to connect us to the metaphysics, and from now on we will be focusing entirely on the math. The first thing to do is to define the stationary action principle; later on we will show how we can get from there to, eventually, everything else.

\section{The Stationary Action Principle}

\[\delta S = 0\]

\subsection{Global Formulation}

The \emph{global action} is defined as the integral of a function called the \emph{Lagrangian}:

\[ S = \int dt L(q, \dot q, t) \]

The assumption of global conformal symmetry (or, equivalently, defining the Lagrangian as a scalar) has a few notable consequences (we will explain how later on):

- The rotation-only part of the global conformal symmetry (i.e. keeping scale constant) results in what we usually call special relativity.

- The scale part of the conformal symmetry, notably, is \emph{broken} in a very specific way. From our perspective, this means that the ``rules change'' as we start looking at smaller distance scales (or larger energy scales). In fact, this happens at a different energy for each of the fundamental forces, with the end result that the farther we zoom in, we seem to go through a series of ``scale phases'' where one force dominates and the others are either too small to matter (and can be renormalized ``into the background'', or often even entirely ignored) or so large that they don't matter in an entirely different way (which is the same reason we don't have to take the sun's gravity into account when we calculate the trajectory of a baseball).

\subsection{Local Formulation}

Local conformal symmetry 

\subsection{Renormalization from Local to Global}

The stationary action principle can be formulated at two ``scales'':

- The
  - The Stationary Action Principle, at Local and Global Scales: 

  - Global and Local Conformal Symmetry with Broken Global Scale Invariance: all terms in the Lagrangian (density) are scalars that are invariant under global and local conformal transformations inside 3+1 spacetime geometry. (This is one way to define ``scalar''; ) The assumption of global rotational symmetry implies special relativity. Local scale symmetry gives rise to an additional rotational degree of freedom in physical laws that we observe as particle ``spin''; with special relativity, we can recover the famous "spin-statistics theorem", which has an extremely important role in linking the local scale to the global scale via the process of renormalization. Global scale invariance is ``broken'' in a very particular way, which allows for the formation of structures like atoms; the fact that we don't know why this should be the case is called the ``hierarchy problem''.
  
  - 

  - Global and local conformal symmetry within 3+1 spacetime geometry, EXCEPT notably that the scale symmetry appears to be broken in a very specific way to enable the nested structure we see today: galaxies, solar systems, atoms, nuclei, quarks. This yields the stress-energy tensor of general relativity, which is where we ``plug in'' the renormalized matter fields. The local conformal symmetry also ``induces'' an additional rotational degree of freedom for particles which we measure as ``spin''. The global conformal symmetry enforces what we see as special relativity, which has a very interesting relationship with spin. Solutions here have discrete parameters, such as particle trajectories (which can be encoded as vectors with 4 components).

  - Global and local gauge symmetry for matter fields. At high energy scales, these gauge symmetries appear to unify. What that overall symmetry group is (the GUT) remains an active area of research; however, so far we have been able to determine that at low energy scales the total gauge symmetry breaks into at least three parts that each operate at specific scales: U(1) electromagnetism, SU(2) weak nuclear, and SU(3) strong nuclear. These solutions are called wavefunctions, which are not directly observable but encode all of that information in a way that will become clear later on. We connect them to observable quantities in four-dimensional spacetime via a process called renormalization, which treats the wavefunction as describing ``weights'' for each of an infinite number of (hypothetical) particle trajectories that all must obey the conformal symmetry in spacetime.

- The No Magic Corollary: *that's it!*
